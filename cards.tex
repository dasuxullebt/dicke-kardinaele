%% -*- coding: utf-8 -*

\documentclass[a4paper,fontsize=11pt]{scrartcl}
\usepackage[utf8]{inputenc}
\usepackage{ngerman}
\usepackage{CJKutf8}
\usepackage{graphicx}
\usepackage{wrapfig}
\usepackage{fancyhdr}
\usepackage{amsfonts}
\usepackage{amsmath}
\usepackage{stmaryrd}
\newenvironment{Japanese}{%  
\CJKfamily{min}%  
\CJKtilde  
\CJKnospace}{}
\pagestyle{fancy}

\newcommand{\rng}{\operatorname{rng}}
\newcommand{\ZFC}{\operatorname{ZFC}}
\newcommand{\Con}{\operatorname{Con}}
\newcommand{\On}{\operatorname{On}}
\newcommand{\cf}{\operatorname{cf}}
\newcommand{\dom}{\operatorname{dom}}

\fancyhf{}
\fancyhead[L]{Große Kardinalzahlen}
\fancyhead[C]{}
\fancyhead[R]{\thepage}
\renewcommand{\headrulewidth}{0.4pt}
%%\author{Prof. Dr. H.-D. Donder}       %Authors
\begin{document}
\flushleft
\section*{Vorwort}
{\bf Vorkenntnisse:} Logik, Modelle der Mengenlehre (Konstruktibles
Universum, Innere Modelle, kein Forcing)

{\bf Inhalt:} Ursprungliche Motivation für groke Kardinalzahl-Axiome
war, dass $\omega$ ein Sprung in den Kardinalzahlen ist. Man überlegte
ob so ein Sprungverhalten öfter vorkommen könnte. Die erste
Vorgehensweise war, bestimmte Eigenschaften von $\omega$ zu isolieren,
und dann eine Zahl mit den gleichen Eigenschaften ungleich $\omega$
gefordert hat. Dies ist allerdings inzwischen weitgehend obsolet.

Die Theorien die man durch große Kardinalzahlen bekommt ist eine
Verstärkung der ursprünglichen Theorie. Man kann mit immer neuen
großen Kardinalzahlen immer neue Aussagen über natürliche Zahlen
beweisen.

Das hier Behandelte ist im Wesentlichen die Einführung immer größerer
Kardinalzahlen, relevant wird immer sein, ob man noch die natürlichen
inneren Modelle für große Kardinalzahlen hat.

{\bf Literatur:}
\begin{itemize}
  \item Drake, Set Theory - An Introduction To Large Cardinals
  \item Jech, Set Theory
  \item Kanamori, The Higher Infinite
\end{itemize}

{\bf Notation:}
\begin{itemize}
  \item $[X]^n$ = Menge aller $n$-Elementigen Teilmengen von $X$
  \item $[X]^{<\omega} = \bigcup_n[X]^n$ = Menge aller endlichen
    Teilmengen von $X$
\end{itemize}

Wir betrachten oft Strukturen der Form
$\left<M,\in,A_1,\ldots,A_n\right>$ mit $A_i\subseteq M$.

{\bf Konvention:} Lasse das $\in$ weg, identifiziere
$\left<M,A_1,\ldots,A_n\right>$ mit
$\left<M,\in,A_1,\ldots,A_n\right>$.

{\bf Elementare Substruktur:} Sei
$\mathfrak{M}=\left<M,A_1,\ldots,A_n\right>$, $N\subseteq M$ und
$\mathfrak{N}=\left<N,A_1\cap N,\ldots,A_n\cap N\right>$ die
zugehörige Substruktur von $\mathfrak{M}$. Dann ist $\mathfrak{N}$
elementare Substruktur von $\mathfrak{M}$
(Bez. $\mathfrak{N}\prec\mathfrak{M}$ genau dann wenn für alle
$\varphi$ und $a_1,\ldots,a_k\in N$
$$ \mathfrak{M}\models \phi[a_1,\ldots,a_k] \mbox{gdw}
\mathfrak{N}\models\phi[a_1,\ldots,a_k]$$

{\bf Konvention:} Schreibe $N\prec\left<M,A_1,\ldots,A_n\right>$ statt
$\left<N,A_1\cap N,\ldots,A_n\cap
N\right>\prec\left<M,A_1,\ldots,A_n\right>$.

{\bf Skolemfunktionen:} Sei
$\mathfrak{M}=\left<M,\in,A_1,\ldots,A_n\right>$ eine
${\mathcal{L}}$-Struktur. Für eine $\mathcal{L}$-Formel
$\varphi(\vec{x},y)$ definiere $f_\varphi$ so, dass gilt

$$ \mathfrak{M}\models \exists_y \varphi[\vec{a},y] \Rightarrow
\mathfrak{M}\models\varphi[\vec{a},f_\varphi(\vec{a})]$$

Setze
$\mathcal{F}=\{f_\phi|\phi\phantom{a}\mathcal{L}-\mbox{Formel}\}$. Dann
gilt:

Sei $N\subseteq M$, $N\neq\emptyset$, unter allen $f\in\mathcal{F}$
abgeschlossen (d.h. $\vec{a}\in N,f\in\mathcal{F}\Rightarrow
f(\vec{a})\in N$). Dann ist $N\prec\mathfrak{M}$.

\underline{Wir sagen:} $\mathcal{F}$ ist Menge von Skolemfunktionen
für $\mathfrak{M}$. Beachte, dass $\mathcal{F}$ abzählbar ist.

\section{``Kleine'' große Kardinalzahlen}
Arbeite in ZFC. Wir nehmen an, dass ZFC widerspruchsfrei ist.

{\bf Definition:} $\kappa$ ist (stark) unerreichbar gdw
$\kappa>\omega$, $\kappa$ regulär und für alle $\lambda<\kappa$ gilt
$2^\lambda<\kappa$.

{\bf Bemerkung:} $\omega$ ist regulär und für alle $n<\omega$ ist
$2^n<\omega$.

{\bf Lemma 1:} Sei $\kappa$ unerreichbar. Dann gilt für alle
$\alpha<\kappa$ dass $|V_\alpha<\kappa|$ (also $|V_\kappa|=\kappa$).

{\bf Beweis:} Durch Induktion über $\alpha<\kappa$:
\begin{itemize}
  \item $\alpha=0$: $V_0=\emptyset$. Also $|V_0|=0<\kappa$.
  \item $\alpha=\beta+1$: $V_{\beta+1}=\mathfrak{P}(V_\beta)$. Also
    $|V_{\beta+1}|=2^{|V_\beta|}$. Nach Induktion ist
    $|V_\beta|<\kappa$. Also nach Definition $2^{|V_\beta|}<\kappa$.
  \item $\lim(\lambda)$: Es ist
    $V_\lambda=\bigcup\limits_{\alpha<\lambda} V_\alpha$. Also
    $|V_\lambda|\le |\lambda|\cdot
    \sup\{|V_\alpha||\alpha<\lambda\}$. Wegen $\lambda<\kappa$ genügt
    es also zu zeigen, dass
    $\sup\{|V_\alpha||\alpha<\lambda\}<\kappa$. Also für
    $\alpha<\lambda$ ist $|V_\alpha|<\kappa$ nach
    Induktionsvoraussetzung und $\lambda<\kappa$. Wegen $\kappa$
    regulär ist also $\sup\{|V_\alpha||\alpha<\lambda\}<\kappa$.
\end{itemize}
\hfill $\square$

{\bf Lemma 2:} Sei $\kappa$ unerreichbar und $u\subseteq
V_\kappa$. Dann gilt $u\in V_\kappa$ gdw $|u|<\kappa$.

{\bf Beweis:}
\begin{itemize}
\item[$\rightarrow$:] Sei $u\in V_\kappa$. Dann existiert
  $\alpha<\kappa$ mit $u\in V_\alpha$. Wegen $V_\alpha$ transitiv also
  $u\subseteq V_\alpha$. Somit $|u|\le|V_\alpha|
  \stackrel{\mbox{\tiny L1}}{<} \kappa$.
\item[$\leftarrow$:] Sei $|u|<\kappa$. Definiere $g:u\rightarrow \On$
  durch $g(x)=\min\{\alpha|x\in V_\alpha\}$. Wegen $u\subseteq V_k$
  ist $\rng(g)\subseteq\kappa$. Aber $|\rng(g)|\le |u|<\kappa$. Wegen
  $\kappa$ regulär ist also $\rng(g)$ beschränkt in $\kappa$. Sei
  $\gamma=\sup \rng (g)$. Dann ist $u\subseteq V_\gamma$, denn für
  $x\in u$ ist $x\in V_{g(x)}\subseteq V_\gamma$. Also $u\in
  V_{\gamma+1}\subseteq V_\kappa$.
\end{itemize}
\hfill $\square$

{\bf Satz 3:} Sei $\kappa$ unerreichbar. Dann $V_\kappa\models ZFC$.

{\bf Beweis:}
\begin{itemize}
\item (Ext) und (Fund) sind $\Pi_1$. Sie gelten also in $V_\kappa$, da
  $V_\kappa$ trasitiv.
\item (Null) da $\emptyset\in V_\kappa$
\item (Un) da $\omega=\On\cap V_\omega \in V_{\omega+1}\subseteq
  V_\kappa$ (da $\kappa>\omega$)
\item (Paar) Seien $x,y\in V_\kappa$. Dann existieren
  $\alpha,\beta<\kappa$ mit $x\in V_\alpha$ und $y\in V_\beta$. Setze
  $\gamma=\max\{\alpha,\beta\}$. Also $\{x,y\}\subseteq V_\gamma$ und
  somit $\{x,y\}\in V_{\gamma+1}\subseteq V_\kappa$.
\item (Ver) Sei $x\in V_\kappa$. Dann existiert $\alpha<\kappa$ mit
  $x\in V_\alpha$. Also wegen $V_\alpha$ transitiv $\bigcup x\subseteq
  V_\alpha$. Somit $\bigcup x\in V_{\alpha+1}\subseteq V_\kappa$.
\item (Pot) Sei $x\in V_\kappa$. Dann existiert $\alpha<\kappa$ mit
  $x\in V_\alpha$. Also $x\subseteq V_\alpha$ und daher
  $\mathfrak{P}(x)\subseteq\mathfrak{P}(V_\alpha)=V_{\alpha+1}$. Somit
  $\mathfrak{P}(X)\in V_{\alpha+2}\subseteq V_\kappa$.
\item (Ers) Es genügt folgendes zu zeigen: Sei $F:A\rightarrow
  V_\kappa$ Funktion mit $A\subseteq V_\kappa$ und sei $u\in
  V_\kappa$. Dann ist $F''u\in V_\kappa$. Sei also $F:A\rightarrow
  V_\kappa$ gegeben und $u\in V_\kappa$. Dann ist auch $u\subseteq
  V_\kappa$ und nach Lemma 2 $|u|<\kappa$. Setze $v=F''u$. Also
  $v\subseteq V_\kappa$ und $|v|\le |u|<\kappa$. Somit nach Lemma 2
  $v\in V_\kappa$.
\item (AC) Sei $u\in V_\kappa$ mit $\emptyset\not\in u$. Sei $f$ eine
  Auswahlfunktion für $u$. Also ist $f\subseteq u\times\bigcup u$. Sei
  $\alpha<\kappa$ mit $u\in V_\alpha$. Dann ist aber $u\times\bigcup
  u\subseteq V_\alpha\times V_\alpha\subseteq V_{\alpha+2}$. Also
  $f\in V_{\alpha+3}\subseteq V_\kappa$.
\end{itemize}
\hfill $\square$

{\bf Folgerung:} $\ZFC + \exists_\kappa \kappa\mbox{ unerreichbar
}\vdash \Con(\ZFC)$. Also nach Gödel
$\ZFC\not\vdash\exists_\kappa\kappa\mbox{ unerreichbar}$. Sogar ohne
Gödel, denn:

{\bf Annahme:} $\ZFC\vdash\exists_\kappa\kappa\mbox{
  unerreichbar}$. Sei $\kappa$ die kleinste unerreichbare
Kardinalzahl. Dann $V_\kappa\models\ZFC+\mbox{existiert keine
  unerreichbare Kardinalzahl}$. Widerspruch.


%% ========= EINGEFÜGT ===============


{\bf Satz 4:} Es sind äquivalent
\begin{itemize}
\item[(1)] $\kappa$ ist unerreichbar
\item[(2)] Für alle $A\subseteq V_{\kappa}$ ist $\{\alpha <\kappa \mid
  V_{\alpha} \prec \langle V_{\kappa}, A\rangle\}$ abgeschlossen und
  unbeschränkt in $\kappa$
\item[(3)] Für alle $A\subseteq V_{\kappa}$ und (erststufigen) $\varphi$
  mit $\langle V_{\kappa}, A\rangle\models \varphi$ existiert
  $\alpha<\kappa$ mit $\langle V_{\alpha}, A\cap V_{\alpha} \rangle
  \models \varphi$
\end{itemize}

{\bf Beweis:}
\begin{itemize}
\item (1)$\Rightarrow$(2): Sei $A\subseteq V_{\kappa}$. Setze $C=\{\alpha<\kappa\mid V_{\alpha}\prec \langle V_{\kappa}, A\rangle\}$.
  \begin{itemize}
  \item[(a)] $C\mbox{ ist abgeschlossen in }\kappa$: Sei $\alpha <
    \kappa$ ein Limespunkt von $C$.  Dann ist $V_{\alpha} =
    \bigcup_{\gamma\in C\cap\alpha} V_{\gamma}$.  Also $V_{\alpha}
    \prec \langle V_{\kappa}, A\rangle$ nach dem Lemma von Tarski über
    elementare Ketten.
  \item[(b)] $C\mbox{ ist unbeschränkt in }\kappa$: Sei $\mathcal F$
    eine Menge von Skolemfunktionen für $\langle V_{\kappa},
    A\rangle$.  Für $f\in \mathcal F$ sei $C_f = \{\alpha<\kappa\mid
    f''V_{\alpha}^{\kappa_f} \subseteq V_{\alpha}\}$.
  \item[(c)] $C_f\mbox{ ist abeschlossen und unbeschränkt in }\kappa$:
    Abgeschlossenheit ist klar.  Zur Unbeschränktheit: Sei
    $\alpha<\kappa$.  Definiere rekursiv die monotone Folge
    $\alpha_0<\alpha_1<\cdots<\kappa$ wie folgt: Setze
    $\alpha_0=\alpha$.  Im $n$-ten Schritt $v_n=
    f''V_{\alpha_n}^{\kappa_f}$.  Dann wegen $\kappa$ unerreichbar
    $|v_n|<\kappa$.  Also ist $v_n\in V_{\kappa}$ nach Lemma 2.  Also
    existiert $\beta>\alpha_n$ mit $\beta<\kappa$ und $v_n\subseteq
    V_{\beta}$.  Setze $\alpha_{n+1}=\beta$.  Dann gilt also
    $f''V_{\alpha_n}^{n_f}\subseteq V_{\alpha_{n+1}}$.  Sei dann
    $\gamma = \sup_n \alpha_n$.  Dann $\gamma <\kappa$, da
    $\cf(\kappa)>\kappa$ und natürlich $\gamma >\alpha$.  Aber
    $\gamma\in C_f$, da
    $f''V_{\gamma}^{n_f}=\bigcup_{n\in\omega}f''V_{\alpha}^{n_f}
    \subseteq \bigcup_{n\in\omega}V_{\alpha_{n+1}} = V_{\gamma}$.
  \end{itemize}
  Setze nun $D=\bigcup_{f\in \mathcal F} C_f$.  Wegen $\mathcal F$
  abzählbar ist $D$ nach $(c)$ abg. unb. in $\kappa$.  Aber
  $D\subseteq C$, dann ist $\alpha\in D$, so für alle $f\in \mathcal
  F$: $f''V^{\kappa_f}_{\alpha}\subseteq V_{\alpha}$, also
  $V_{\alpha}\prec \langle V_{\kappa}, A\rangle$.
\item (2)$\Rightarrow$(3): Trivial.
\item (3)$\Rightarrow$(1): Sei (3) erfüllt.
  \begin{itemize}
  \item[(i)] $\kappa\mbox{ ist Limesordinalzahl}$: Offenbar
    $\kappa\neq 0$.  $\lightning$-Annahme: $\kappa=\gamma+1$.  Setze
    $A=\{\gamma\}$.  Dann $\langle V_{\kappa},A\rangle\models
    A\neq\emptyset$ (genau: $\langle V_{\kappa},A\rangle\models
    \exists x\; A(x)$). Wähle nach (3) ein $\alpha<\kappa$ mit
    $\langle V_{\alpha}, A\cap V_{\alpha}\rangle\models A\neq
    \emptyset$.  Dann aber $\gamma \in A\cap V_{\alpha}$, d.h. $\gamma
    \in V_{\alpha}$, also $\gamma<\alpha$.  $\lightning$ zu
    $\alpha<\kappa=\gamma+1$.
  \item[(ii)] $\kappa>\omega$: $\lightning$-Ann.: $\kappa<\omega$.
    Dann nach $(i)$ $\kappa=\omega$.  Aber
    $(V_{\omega}\models\forall\gamma\in\On\; \exists \delta\in\On\;
    \gamma<\delta)\land \exists x\; x=x$ %underbrace \varphi und für
    alle $n<\omega$: $V_n\models \neg\varphi$.  $\lightning$ zu (3).
  \item[(iii)] $\kappa\mbox{ ist regulär}$: Sei $\gamma<\kappa$ und
    $f\colon\gamma\to\kappa$.  Wir müssen zeigen, dass $f$ beschränkt
    in $\kappa$ ist.  Wir können o.E. annehmen, dass $f(0)=\gamma$.
    Setze $A=f\subseteq V_{\kappa}$.  Dann \[\langle
    V_{\kappa},A\rangle\models \mbox{''$A$ ist Funktion, dann
      $(A)=A(0)$''}\] Gemäß (3) wähle $\alpha<\kappa$ mit $\langle
    V_{\alpha}, A\cap V_{\alpha}\models \varphi$.  Setze $g=A\cap
    V_{\alpha}$.  Dann $0\in\dom(g)$.  Also $g(0)=f(0)=\gamma \in
    V_{\alpha}$.  Außerdem $dom(g)=g(0)=\gamma$.  Also $g=f$,
    d.h. $f\subseteq V_{\alpha}$ und damit $\rng(f)\subseteq On\cap
    V_{\alpha}=\alpha$.  Also $f$ beschränkt in $\kappa$.
  \item[(iv)] $\mbox{für alle} \lambda<\kappa \; 2^{\lambda}<\kappa$:
    Sei $\lambda<\kappa$ und $f\colon \mathfrak P(\lambda)\to\kappa$.
    Wir müssen zeigen, dass $f$ nicht surjektiv ist.  Sei
    o.E. $f(\emptyset)=\lambda$.  Setze $A=f\subseteq V_{\kappa}$.
    Dann $\langle V_{\kappa}, A\rangle\models \mbox{''$A$ ist
      Funktion, $\dom(A)=\mathfrak(A(\emptyset))$''}$.  Gemäß (3)
    wähle $\alpha<\kappa$ mit $\langle V_{\alpha}, A\cap
    V_{\alpha}\rangle\models\varphi$.  Setze $g=A\cap V_{\alpha}$.
    Dann $\emptyset\in\dom(g)$.  Also
    $g(\emptyset)=f(\emptyset)=\lambda\in V_{\alpha}$.  Außerdem
    $\dom(g)=\mathfrak P(g(\emptyset)) = \mathfrak P(\lambda)$.  Also
    $g=f$, d.h. $f\subseteq V_{\alpha}$ und somit
    z.B. $\alpha\not\in\rng(f)$
  \end{itemize}
\end{itemize}

{\bf Lemma 5:} Sei $W$ ein inneres Modell. Sei $\kappa$
unerreichbar. Dann ist $\kappa$ unerreichbar in $W$.

{\bf Beweis:} $\kappa$ ist regulär in $W$, denn Regularität ist eine
$\Pi_1$-Eigenschaft. Sei $\lambda<\kappa$. Dann $(2^{\lambda})^W\le
2^{\lambda}<\kappa$.

{\bf Definition:} $\kappa$ ist \emph{schwach unerreichbar} $\iff$
$\kappa>\omega$ und $\kappa$ ist reguläre Limesordinalzahl.

{\bf Anmerkung:} Sei $W$ ein inneres Modell. Sei $\kappa$ schwach
unerreichbar. Dann ist $\kappa$ schwach unerreichbar in $W$.

{\bf Anmerkung:} Sei $\kappa$ schwach unerreichbar. Dann ist $\kappa$ unerreichbar in $L$.

{\bf Beweis:} $\kappa$ ist schwach unerreichbar in $L$. Aber $L\models
GCH$. Also ist $\kappa$ unerreichbar in $L$.

Wir können z.B. definieren: $\kappa$ ist \emph{hyperunerreichbar}
$\iff$ $\kappa$ ist unerreichbar und $\kappa
=\sup\{\tau<\kappa\mid\tau \mbox{ unerreichbar}\}$. Das lässt sich
beliebig weitertreiben: hyperhyperunerreichbar,
hyperhyperhyperunerreichbar usw...

{\bf Definition:} $\kappa$ ist \emph{Mahlo} $\iff$ $\kappa$ ist
unerreichbar und $\{\tau<\kappa\mid \tau \mbox{ ist regulär}\}$ ist
stationär in $\kappa$.

{\bf Bemerkung:} $\kappa$ Mahlo $\Rightarrow$ $\{ \tau<\kappa\mid \tau\mbox{ unerreichbar}\}$ ist stationär in $\kappa$.

{\bf Beweis:} Sei $\kappa$ Mahlo. Sei $E= \{\tau<\kappa\mid \tau
\mbox{ regulär} \}$. Setze \[C=\{ \tau<\kappa\mid \tau>\omega \mbox{
  und für alle }\lambda<\tau\; 2^{\lambda}<\tau\}\]. $C$ ist
abg. unb. in $\kappa$, denn: Abg. ist klar.  Unb.: Sei
$\alpha<\kappa$.  Setze $\tau_0=\max\{\alpha,\omega\}$,
$\tau_{n+1}=2^{\tau_n}$.  Wegen $\kappa$ unerreichbar ist
$\tau_n<\kappa$.  Setze $\tau=\sup_n\tau_n$.  Dann auch $\tau<\kappa$
und $\tau\in C$.  Wegen $\kappa$ Mahlo ist $E$ stationär in $\kappa$.
Also ist $\{\tau<\kappa\mid \tau\mbox{ unerreichbar }\} = C\cap E$
stationär in $\kappa$.

{\bf Lemma 6:} Sei $\kappa$ Mahlo. Dann gilt: 
\begin{itemize}
\item[(a)] $\kappa$ ist hyperunerreichbar
\item[(b)] $\{\tau<\kappa\mid \tau \mbox{ ist hyperunerreichbar}\}$ ist stationär in $\kappa$.
\end{itemize}

{\bf Beweis:}
\begin{itemize}
\item[(a)] Ist klar, da nach Bemerkung $\{\tau<\kappa\mid \tau\mbox{ unerreichbar}\}$  sogar stationär in $\kappa$, also auch unbeschränkt in $\kappa$.
\item[(b)] Sei $A=\{\tau<\kappa\mid \tau \mbox{ unerreichbar}\}$. Nach
  (a) ist $A$ unbeschränkt in $\kappa$. Also ist $C=\{ \tau<\kappa\mid
  \tau=\sup(A\cap\tau)\}$ abg. unb. in $\kappa$. Da $A$ sogar
  stationär in $\kappa$, ist also $\{\tau<\kappa\mid \kappa \mbox{ ist
    hyperunerreichbar}\}=C\cap A$ stationär in $\kappa$.
\end{itemize}

%% ========= /EINGEFÜGT ===============

{\bf 1. Übung:}

{\bf Aufgabe 1:} Sei $\kappa\in\On$. Zeigen Sie, dass folgende
Aussagen äquivalent sind:
\begin{enumerate}
\item $\kappa$ ist unerreichbar
\item für alle $f:V_\kappa\rightarrow\kappa$ existiert
  $0<\alpha<\kappa$ mit $f''V_\alpha\subseteq\alpha$
\end{enumerate}

{\bf Aufgabe 2:} Seien $\alpha,\beta\in\On$ mit $\alpha<\beta$, und es
gelte $V\alpha\prec V_\beta$. Zeigen Sie, dass $V_\alpha\models\ZFC$.

\end{document}
