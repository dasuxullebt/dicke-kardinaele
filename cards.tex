%% -*- coding: utf-8 -*

\documentclass[a4paper,fontsize=11pt]{scrartcl}
\usepackage[utf8]{inputenc}
\usepackage{ngerman}
\usepackage{CJKutf8}
\usepackage{graphicx}
\usepackage{wrapfig}
\usepackage{fancyhdr}
\usepackage{amsfonts}
\usepackage{amsmath}
\usepackage{amssymb}
\usepackage{stmaryrd}
\usepackage{hyperref}
\usepackage{undertilde}
\usepackage{setspace}
\newenvironment{Japanese}{%  
\CJKfamily{min}%  
\CJKtilde  
\CJKnospace}{}
\pagestyle{fancy}
\usepackage{tikz}
\usetikzlibrary{arrows}% For nice arrow tips
% One style for all TikZ pictures for working with overlays:
\tikzset{every picture/.style=remember picture}
% Define a TikZ node for math content:
\newcommand{\mathnode}[2]{%
  \mathord{\tikz[baseline=(#1.base), inner sep = 0pt]{\node (#1) {$#2$};}}}

\newcommand{\rng}{\operatorname{rng}}
\newcommand{\ZFC}{\operatorname{ZFC}}
\newcommand{\Con}{\operatorname{Con}}
\newcommand{\On}{\operatorname{On}}
\newcommand{\cf}{\operatorname{cf}}
\newcommand{\otp}{\operatorname{otp}}
\newcommand{\dom}{\operatorname{dom}}
\newcommand{\id}{\operatorname{id}}
\newcommand{\height}{\operatorname{ht}}
\renewcommand{\thefootnote}{[siehe \arabic{footnote}]}

\fancyhf{}
\fancyhead[L]{Große Kardinalzahlen}
\fancyhead[C]{}
\fancyhead[R]{\thepage}
\renewcommand{\headrulewidth}{0.4pt}
%%\author{Prof. Dr. H.-D. Donder}       %Authors
\begin{document}
\flushleft
\section*{Vorwort}
Fehler und Anmerkungen bitte an \textsc{christoph {\it punkt} senjak {\it at} ifi {\it punkt} lmu {\it punkt} de}.

{\bf Vorkenntnisse:} Logik, Modelle der Mengenlehre (Konstruktibles
Universum, Innere Modelle, kein Forcing)

{\bf Inhalt:} Ursprungliche Motivation für groke Kardinalzahl-Axiome
war, dass $\omega$ ein Sprung in den Kardinalzahlen ist. Man überlegte
ob so ein Sprungverhalten öfter vorkommen könnte. Die erste
Vorgehensweise war, bestimmte Eigenschaften von $\omega$ zu isolieren,
und dann eine Zahl mit den gleichen Eigenschaften ungleich $\omega$
gefordert hat. Dies ist allerdings inzwischen weitgehend obsolet.

Die Theorien die man durch große Kardinalzahlen bekommt ist eine
Verstärkung der ursprünglichen Theorie. Man kann mit immer neuen
großen Kardinalzahlen immer neue Aussagen über natürliche Zahlen
beweisen.

Das hier Behandelte ist im Wesentlichen die Einführung immer größerer
Kardinalzahlen, relevant wird immer sein, ob man noch die natürlichen
inneren Modelle für große Kardinalzahlen hat.

{\bf Literatur:}
\begin{itemize}
  \item Drake, Set Theory - An Introduction To Large Cardinals
  \item Jech, Set Theory
  \item Kanamori, The Higher Infinite
\end{itemize}

{\bf Notation und Notizen:}
\begin{itemize}
  \item $[X]^n$ = Menge aller $n$-Elementigen Teilmengen von $X$
  \item $[X]^{<\omega} = \bigcup_n[X]^n$ = Menge aller endlichen
    Teilmengen von $X$
  \item gdw: genau dann wenn
  \item abg. unb.: abgeschlossen unbeschränkt
  \item $\otp$: Ordnungstyp
  \item $\overline{\cdot}$ ist {\bf kein} operator, sondern wird
    benutzt um zusätzliche Variablennamen zu generieren,
    $\overline{\kappa}$ hängt nicht mit $\kappa$ zusammen.
  \item $j:M\rightarrow_{\Sigma_\omega} N$: $j$ ist elementare
    Einbettung von $M$ in $N$
\end{itemize}

Wir betrachten oft Strukturen der Form
$\left<M,\in,A_1,\ldots,A_n\right>$ mit $A_i\subseteq M$.

{\bf Konvention:} Lasse das $\in$ weg, identifiziere
$\left<M,A_1,\ldots,A_n\right>$ mit
$\left<M,\in,A_1,\ldots,A_n\right>$.

{\bf Elementare Substruktur:} Sei
$\mathfrak{M}=\left<M,A_1,\ldots,A_n\right>$, $N\subseteq M$ und
$\mathfrak{N}=\left<N,A_1\cap N,\ldots,A_n\cap N\right>$ die
zugehörige Substruktur von $\mathfrak{M}$. Dann ist $\mathfrak{N}$
elementare Substruktur von $\mathfrak{M}$
(Bez. $\mathfrak{N}\prec\mathfrak{M}$ genau dann wenn für alle
$\varphi$ und $a_1,\ldots,a_k\in N$
$$ \mathfrak{M}\models \varphi[a_1,\ldots,a_k] \Leftrightarrow
\mathfrak{N}\models\varphi[a_1,\ldots,a_k]$$

{\bf Konvention:} Schreibe $N\prec\left<M,A_1,\ldots,A_n\right>$ statt
$\left<N,A_1\cap N,\ldots,A_n\cap
N\right>\prec\left<M,A_1,\ldots,A_n\right>$.

{\bf Skolemfunktionen:} Sei
$\mathfrak{M}=\left<M,\in,A_1,\ldots,A_n\right>$ eine
${\mathcal{L}}$-Struktur. Für eine $\mathcal{L}$-Formel
$\varphi(\vec{x},y)$ definiere $f_\varphi$ so, dass gilt

$$ \mathfrak{M}\models \exists_y \varphi[\vec{a},y] \Rightarrow
\mathfrak{M}\models\varphi[\vec{a},f_\varphi(\vec{a})]$$

Setze
$\mathcal{F}=\{f_\varphi|\varphi\phantom{a}\mathcal{L}-\mbox{Formel}\}$. Dann
gilt:

Sei $N\subseteq M$, $N\neq\emptyset$, unter allen $f\in\mathcal{F}$
abgeschlossen (d.h. $\vec{a}\in N,f\in\mathcal{F}\Rightarrow
f(\vec{a})\in N$). Dann ist $N\prec\mathfrak{M}$.

\underline{Wir sagen:} $\mathcal{F}$ ist Menge von Skolemfunktionen
für $\mathfrak{M}$. Beachte, dass $\mathcal{F}$ abzählbar ist.

\section{``Kleine'' große Kardinalzahlen}
Arbeite in ZFC. Wir nehmen an, dass ZFC widerspruchsfrei ist.

{\bf Definition:} $\kappa$ ist (stark) unerreichbar $\Leftrightarrow$
$\kappa>\omega$, $\kappa$ regulär und für alle $\lambda<\kappa$ gilt
$2^\lambda<\kappa$.

{\bf Bemerkung:} $\omega$ ist regulär und für alle $n<\omega$ ist
$2^n<\omega$.

{\bf Lemma 1:} Sei $\kappa$ unerreichbar. Dann gilt für alle
$\alpha<\kappa$ dass $|V_\alpha|<\kappa$ (also $|V_\kappa|=\kappa$).

{\bf Beweis:} Durch Induktion über $\alpha<\kappa$:
\begin{itemize}
  \item $\alpha=0$: $V_0=\emptyset$. Also $|V_0|=0<\kappa$.
  \item $\alpha=\beta+1$: $V_{\beta+1}=\mathfrak{P}(V_\beta)$. Also
    $|V_{\beta+1}|=2^{|V_\beta|}$. Nach Induktion ist
    $|V_\beta|<\kappa$. Also nach Definition $2^{|V_\beta|}<\kappa$.
  \item $\lim(\lambda)$: Es ist
    $V_\lambda=\bigcup\limits_{\alpha<\lambda} V_\alpha$. Also
    $|V_\lambda|\le |\lambda|\cdot
    \sup\{|V_\alpha||\alpha<\lambda\}$. Wegen $\lambda<\kappa$ genügt
    es also zu zeigen, dass
    $\sup\{|V_\alpha||\alpha<\lambda\}<\kappa$. Also für
    $\alpha<\lambda$ ist $|V_\alpha|<\kappa$ nach
    Induktionsvoraussetzung und $\lambda<\kappa$. Wegen $\kappa$
    regulär ist also $\sup\{|V_\alpha||\alpha<\lambda\}<\kappa$.
\end{itemize}
\hfill $\square$

{\bf Lemma 2:} Sei $\kappa$ unerreichbar und $u\subseteq
V_\kappa$. Dann gilt $u\in V_\kappa\Leftrightarrow|u|<\kappa$.

{\bf Beweis:}
\begin{itemize}
\item[$\rightarrow$:] Sei $u\in V_\kappa$. Dann existiert
  $\alpha<\kappa$ mit $u\in V_\alpha$. Wegen $V_\alpha$ transitiv also
  $u\subseteq V_\alpha$. Somit $|u|\le|V_\alpha|
  \stackrel{\mbox{\tiny L1}}{<} \kappa$.
\item[$\leftarrow$:] Sei $|u|<\kappa$. Definiere $g:u\rightarrow \On$
  durch $g(x)=\min\{\alpha|x\in V_\alpha\}$. Wegen $u\subseteq V_k$
  ist $\rng(g)\subseteq\kappa$. Aber $|\rng(g)|\le |u|<\kappa$. Wegen
  $\kappa$ regulär ist also $\rng(g)$ beschränkt in $\kappa$. Sei
  $\gamma=\sup \rng (g)$. Dann ist $u\subseteq V_\gamma$, denn für
  $x\in u$ ist $x\in V_{g(x)}\subseteq V_\gamma$. Also $u\in
  V_{\gamma+1}\subseteq V_\kappa$.
\end{itemize}
\hfill $\square$

{\bf Satz 3:} Sei $\kappa$ unerreichbar. Dann $V_\kappa\models ZFC$.

{\bf Beweis:}
\begin{itemize}
\item (Ext) und (Fund) sind $\Pi_1$. Sie gelten also in $V_\kappa$, da
  $V_\kappa$ trasitiv.
\item (Null) da $\emptyset\in V_\kappa$
\item (Un) da $\omega=\On\cap V_\omega \in V_{\omega+1}\subseteq
  V_\kappa$ (da $\kappa>\omega$)
\item (Paar) Seien $x,y\in V_\kappa$. Dann existieren
  $\alpha,\beta<\kappa$ mit $x\in V_\alpha$ und $y\in V_\beta$. Setze
  $\gamma=\max\{\alpha,\beta\}$. Also $\{x,y\}\subseteq V_\gamma$ und
  somit $\{x,y\}\in V_{\gamma+1}\subseteq V_\kappa$.
\item (Ver) Sei $x\in V_\kappa$. Dann existiert $\alpha<\kappa$ mit
  $x\in V_\alpha$. Also wegen $V_\alpha$ transitiv $\bigcup x\subseteq
  V_\alpha$. Somit $\bigcup x\in V_{\alpha+1}\subseteq V_\kappa$.
\item (Pot) Sei $x\in V_\kappa$. Dann existiert $\alpha<\kappa$ mit
  $x\in V_\alpha$. Also $x\subseteq V_\alpha$ und daher
  $\mathfrak{P}(x)\subseteq\mathfrak{P}(V_\alpha)=V_{\alpha+1}$. Somit
  $\mathfrak{P}(X)\in V_{\alpha+2}\subseteq V_\kappa$.
\item (Ers) Es genügt folgendes zu zeigen: Sei $F:A\rightarrow
  V_\kappa$ Funktion mit $A\subseteq V_\kappa$ und sei $u\in
  V_\kappa$. Dann ist $F''u\in V_\kappa$. Sei also $F:A\rightarrow
  V_\kappa$ gegeben und $u\in V_\kappa$. Dann ist auch $u\subseteq
  V_\kappa$ und nach Lemma 2 $|u|<\kappa$. Setze $v=F''u$. Also
  $v\subseteq V_\kappa$ und $|v|\le |u|<\kappa$. Somit nach Lemma 2
  $v\in V_\kappa$.
\item (AC) Sei $u\in V_\kappa$ mit $\emptyset\not\in u$. Sei $f$ eine
  Auswahlfunktion für $u$. Also ist $f\subseteq u\times\bigcup u$. Sei
  $\alpha<\kappa$ mit $u\in V_\alpha$. Dann ist aber $u\times\bigcup
  u\subseteq V_\alpha\times V_\alpha\subseteq V_{\alpha+2}$. Also
  $f\in V_{\alpha+3}\subseteq V_\kappa$.
\end{itemize}
\hfill $\square$

{\bf Folgerung:} $\ZFC + \exists_\kappa \kappa\mbox{ unerreichbar
}\vdash \Con(\ZFC)$. Also nach Gödel
$\ZFC\not\vdash\exists_\kappa\kappa\mbox{ unerreichbar}$. Sogar ohne
Gödel, denn:

{\bf Annahme:} $\ZFC\vdash\exists_\kappa\kappa\mbox{
  unerreichbar}$. Sei $\kappa$ die kleinste unerreichbare
Kardinalzahl. Dann $V_\kappa\models\ZFC+\mbox{existiert keine
  unerreichbare Kardinalzahl}$. Widerspruch.


%% ========= EINGEFÜGT ===============


{\bf Satz 4:} Es sind äquivalent
\begin{itemize}
\item[(1)] $\kappa$ ist unerreichbar
\item[(2)] Für alle $A\subseteq V_{\kappa}$ ist $\{\alpha <\kappa \mid
  V_{\alpha} \prec \langle V_{\kappa}, A\rangle\}$ abgeschlossen und
  unbeschränkt in $\kappa$
\item[(3)] Für alle $A\subseteq V_{\kappa}$ und (erststufigen) $\varphi$
  mit $\langle V_{\kappa}, A\rangle\models \varphi$ existiert
  $\alpha<\kappa$ mit $\langle V_{\alpha}, A\cap V_{\alpha} \rangle
  \models \varphi$
\end{itemize}

{\bf Beweis:}
\begin{itemize}
\item (1)$\Rightarrow$(2): Sei $A\subseteq V_{\kappa}$. Setze $C=\{\alpha<\kappa\mid V_{\alpha}\prec \langle V_{\kappa}, A\rangle\}$.
  \begin{itemize}
  \item[(a)] $C\mbox{ ist abgeschlossen in }\kappa$: Sei $\alpha <
    \kappa$ ein Limespunkt von $C$.  Dann ist $V_{\alpha} =
    \bigcup_{\gamma\in C\cap\alpha} V_{\gamma}$.  Also $V_{\alpha}
    \prec \langle V_{\kappa}, A\rangle$ nach dem Lemma von Tarski über
    elementare Ketten.
  \item[(b)] $C\mbox{ ist unbeschränkt in }\kappa$: Sei $\mathcal F$
    eine Menge von Skolemfunktionen für $\langle V_{\kappa},
    A\rangle$.  Für $f\in \mathcal F$ sei $C_f = \{\alpha<\kappa\mid
    f''V_{\alpha}^{\kappa_f} \subseteq V_{\alpha}\}$.
  \item[(c)] $C_f\mbox{ ist abeschlossen und unbeschränkt in }\kappa$:
    Abgeschlossenheit ist klar.  Zur Unbeschränktheit: Sei
    $\alpha<\kappa$.  Definiere rekursiv die monotone Folge
    $\alpha_0<\alpha_1<\cdots<\kappa$ wie folgt: Setze
    $\alpha_0=\alpha$.  Im $n$-ten Schritt $v_n=
    f''V_{\alpha_n}^{\kappa_f}$.  Dann wegen $\kappa$ unerreichbar
    $|v_n|<\kappa$.  Also ist $v_n\in V_{\kappa}$ nach Lemma 2.  Also
    existiert $\beta>\alpha_n$ mit $\beta<\kappa$ und $v_n\subseteq
    V_{\beta}$.  Setze $\alpha_{n+1}=\beta$.  Dann gilt also
    $f''V_{\alpha_n}^{n_f}\subseteq V_{\alpha_{n+1}}$.  Sei dann
    $\gamma = \sup_n \alpha_n$.  Dann $\gamma <\kappa$, da
    $\cf(\kappa)>\kappa$ und natürlich $\gamma >\alpha$.  Aber
    $\gamma\in C_f$, da
    $f''V_{\gamma}^{n_f}=\bigcup_{n\in\omega}f''V_{\alpha}^{n_f}
    \subseteq \bigcup_{n\in\omega}V_{\alpha_{n+1}} = V_{\gamma}$.
  \end{itemize}
  Setze nun $D=\bigcup_{f\in \mathcal F} C_f$.  Wegen $\mathcal F$
  abzählbar ist $D$ nach $(c)$ abg. unb. in $\kappa$.  Aber
  $D\subseteq C$, dann ist $\alpha\in D$, so für alle $f\in \mathcal
  F$: $f''V^{\kappa_f}_{\alpha}\subseteq V_{\alpha}$, also
  $V_{\alpha}\prec \langle V_{\kappa}, A\rangle$.
\item (2)$\Rightarrow$(3): Trivial.
\item (3)$\Rightarrow$(1): Sei (3) erfüllt.
  \begin{itemize}
  \item[(i)] $\kappa\mbox{ ist Limesordinalzahl}$: Offenbar
    $\kappa\neq 0$.  $\lightning$-Annahme: $\kappa=\gamma+1$.  Setze
    $A=\{\gamma\}$.  Dann $\langle V_{\kappa},A\rangle\models
    A\neq\emptyset$ (genau: $\langle V_{\kappa},A\rangle\models
    \exists x\; A(x)$). Wähle nach (3) ein $\alpha<\kappa$ mit
    $\langle V_{\alpha}, A\cap V_{\alpha}\rangle\models A\neq
    \emptyset$.  Dann aber $\gamma \in A\cap V_{\alpha}$, d.h. $\gamma
    \in V_{\alpha}$, also $\gamma<\alpha$.  $\lightning$ zu
    $\alpha<\kappa=\gamma+1$.
  \item[(ii)] $\kappa>\omega$: $\lightning$-Ann.: $\kappa<\omega$.
    Dann nach $(i)$ $\kappa=\omega$.  Aber
    $(V_{\omega}\models\forall\gamma\in\On\; \exists \delta\in\On\;
    \gamma<\delta)\land \exists x\; x=x$ %underbrace \varphi und für
    alle $n<\omega$: $V_n\models \neg\varphi$.  $\lightning$ zu (3).
  \item[(iii)] $\kappa\mbox{ ist regulär}$: Sei $\gamma<\kappa$ und
    $f\colon\gamma\to\kappa$.  Wir müssen zeigen, dass $f$ beschränkt
    in $\kappa$ ist.  Wir können o.E. annehmen, dass $f(0)=\gamma$.
    Setze $A=f\subseteq V_{\kappa}$.  Dann \[\langle
    V_{\kappa},A\rangle\models \mbox{''$A$ ist Funktion, dann
      $(A)=A(0)$''}\] Gemäß (3) wähle $\alpha<\kappa$ mit $\langle
    V_{\alpha}, A\cap V_{\alpha}\models \varphi$.  Setze $g=A\cap
    V_{\alpha}$.  Dann $0\in\dom(g)$.  Also $g(0)=f(0)=\gamma \in
    V_{\alpha}$.  Außerdem $dom(g)=g(0)=\gamma$.  Also $g=f$,
    d.h. $f\subseteq V_{\alpha}$ und damit $\rng(f)\subseteq On\cap
    V_{\alpha}=\alpha$.  Also $f$ beschränkt in $\kappa$.
  \item[(iv)] $\mbox{für alle} \lambda<\kappa \; 2^{\lambda}<\kappa$:
    Sei $\lambda<\kappa$ und $f\colon \mathfrak P(\lambda)\to\kappa$.
    Wir müssen zeigen, dass $f$ nicht surjektiv ist.  Sei
    o.E. $f(\emptyset)=\lambda$.  Setze $A=f\subseteq V_{\kappa}$.
    Dann $\langle V_{\kappa}, A\rangle\models \mbox{''$A$ ist
      Funktion, $\dom(A)=\mathfrak(A(\emptyset))$''}$.  Gemäß (3)
    wähle $\alpha<\kappa$ mit $\langle V_{\alpha}, A\cap
    V_{\alpha}\rangle\models\varphi$.  Setze $g=A\cap V_{\alpha}$.
    Dann $\emptyset\in\dom(g)$.  Also
    $g(\emptyset)=f(\emptyset)=\lambda\in V_{\alpha}$.  Außerdem
    $\dom(g)=\mathfrak P(g(\emptyset)) = \mathfrak P(\lambda)$.  Also
    $g=f$, d.h. $f\subseteq V_{\alpha}$ und somit
    z.B. $\alpha\not\in\rng(f)$
  \end{itemize}
\end{itemize}
\hfill $\square$

{\bf Lemma 5:} Sei $W$ ein inneres Modell. Sei $\kappa$
unerreichbar. Dann ist $\kappa$ unerreichbar in $W$.

{\bf Beweis:} $\kappa$ ist regulär in $W$, denn Regularität ist eine
$\Pi_1$-Eigenschaft. Sei $\lambda<\kappa$. Dann $(2^{\lambda})^W\le
2^{\lambda}<\kappa$.

\hfill $\square$

{\bf Definition:} $\kappa$ ist \emph{schwach unerreichbar} $\iff$
$\kappa>\omega$ und $\kappa$ ist reguläre Limesordinalzahl.

{\bf Anmerkung:} Sei $W$ ein inneres Modell. Sei $\kappa$ schwach
unerreichbar. Dann ist $\kappa$ schwach unerreichbar in $W$.

{\bf Anmerkung:} Sei $\kappa$ schwach unerreichbar. Dann ist $\kappa$ unerreichbar in $L$.

{\bf Beweis:} $\kappa$ ist schwach unerreichbar in $L$. Aber $L\models
GCH$. Also ist $\kappa$ unerreichbar in $L$.

\hfill $\square$

Wir können z.B. definieren: $\kappa$ ist \emph{hyperunerreichbar}
$\iff$ $\kappa$ ist unerreichbar und $\kappa
=\sup\{\tau<\kappa\mid\tau \mbox{ unerreichbar}\}$. Das lässt sich
beliebig weitertreiben: hyperhyperunerreichbar,
hyperhyperhyperunerreichbar usw...

{\bf Definition:} $\kappa$ ist \emph{Mahlo} $\iff$ $\kappa$ ist
unerreichbar und $\{\tau<\kappa\mid \tau \mbox{ ist regulär}\}$ ist
stationär in $\kappa$.

{\bf Bemerkung:} $\kappa$ Mahlo $\Rightarrow$ $\{ \tau<\kappa\mid \tau\mbox{ unerreichbar}\}$ ist stationär in $\kappa$.

{\bf Beweis:} Sei $\kappa$ Mahlo. Sei $E= \{\tau<\kappa\mid \tau
\mbox{ regulär} \}$. Setze \[C=\{ \tau<\kappa\mid \tau>\omega \mbox{
  und für alle }\lambda<\tau\; 2^{\lambda}<\tau\}\]. $C$ ist
abg. unb. in $\kappa$, denn: Abg. ist klar.  Unb.: Sei
$\alpha<\kappa$.  Setze $\tau_0=\max\{\alpha,\omega\}$,
$\tau_{n+1}=2^{\tau_n}$.  Wegen $\kappa$ unerreichbar ist
$\tau_n<\kappa$.  Setze $\tau=\sup_n\tau_n$.  Dann auch $\tau<\kappa$
und $\tau\in C$.  Wegen $\kappa$ Mahlo ist $E$ stationär in $\kappa$.
Also ist $\{\tau<\kappa\mid \tau\mbox{ unerreichbar }\} = C\cap E$
stationär in $\kappa$.

\hfill $\square$

{\bf Lemma 6:} Sei $\kappa$ Mahlo. Dann gilt: 
\begin{itemize}
\item[(a)] $\kappa$ ist hyperunerreichbar
\item[(b)] $\{\tau<\kappa\mid \tau \mbox{ ist hyperunerreichbar}\}$ ist stationär in $\kappa$.
\end{itemize}

{\bf Beweis:}
\begin{itemize}
\item[(a)] Ist klar, da nach Bemerkung $\{\tau<\kappa\mid \tau\mbox{ unerreichbar}\}$  sogar stationär in $\kappa$, also auch unbeschränkt in $\kappa$.
\item[(b)] Sei $A=\{\tau<\kappa\mid \tau \mbox{ unerreichbar}\}$. Nach
  (a) ist $A$ unbeschränkt in $\kappa$. Also ist $C=\{ \tau<\kappa\mid
  \tau=\sup(A\cap\tau)\}$ abg. unb. in $\kappa$. Da $A$ sogar
  stationär in $\kappa$, ist also $\{\tau<\kappa\mid \kappa \mbox{ ist
    hyperunerreichbar}\}=C\cap A$ stationär in $\kappa$.
\end{itemize}

\hfill $\square$

%% ========= /EINGEFÜGT ===============

{\bf 1. Übung:}

{\bf Aufgabe 1:} Sei $\kappa\in\On$. Zeigen Sie, dass folgende
Aussagen äquivalent sind:
\begin{enumerate}
\item $\kappa$ ist unerreichbar
\item für alle $f:V_\kappa\rightarrow\kappa$ existiert
  $0<\alpha<\kappa$ mit $f''V_\alpha\subseteq\alpha$
\end{enumerate}

{\bf Aufgabe 2:} Seien $\alpha,\beta\in\On$ mit $\alpha<\beta$, und es
gelte $V\alpha\prec V_\beta$. Zeigen Sie, dass $V_\alpha\models\ZFC$.

{\bf Satz 7:} Es sind äquivalent
\begin{itemize}
  \item[(1)] $\kappa$ ist Mahlo
  \item[(2)] für alle $A\subseteq V_\kappa$ existiert ein reguläres
    $\alpha<\kappa$ mit $V_\alpha\prec\left<V_\kappa,A\right>$
  \item[(3)] für alle $A\subseteq V_\kappa$ mit
    $\left<V_\kappa,A\right>\models\varphi$ ($\varphi$ erststufig)
    existiert ein reguläres $\alpha<\kappa$ mit $\left<V_\alpha,A\cap
    V_\alpha\right>\models\varphi$
\end{itemize}

{\bf Beweis:}
\begin{itemize}
\item (1)$\Rightarrow$(2): Sei $\kappa$ Mahlo. Sei $A\subseteq
  V_\kappa$. Wegen $\kappa$ unerreichbar ist dann nach Satz 4
  $C=\{\alpha<\kappa|V_\alpha\prec\left<V_\kappa,A\right>\}$
    abgeschlossen unbeschränkt in $\kappa$. Wegen $\kappa$ Mahlo
    existiert also $\alpha\in C$ mit $\alpha$ regulär.
\item (2)$\Rightarrow$(3): Trivial.
\item (3)$\Rightarrow$(1): Sei (3) erfüllt. Dann ist $\kappa$
  unerreichbar nach Satz 4. Sei $E=\{\alpha<\kappa|\alpha\mbox{
    regulär}\}$. Noch zu zeigen, dass $E$ stationär in $\kappa$. Sei
  hierzu $C\subseteq\kappa$ abgeschlossen unbeschränkt in
  $\kappa$. Dann
  $\left<V_\kappa,C\right>\models\underbrace{\forall_{\gamma\in\On}\exists_{\delta\in
      C}\gamma<\delta}_\varphi$. Gemäß (3) wähle ein reguläres
  $\alpha<\kappa$ mit $\left<V_\alpha,C\cap
  V_\alpha\right>\models\varphi$. Nun ist $C\cap V_\alpha=C\cap\alpha$
  da $C\subseteq\On$ und $V_\alpha\cap\On=\alpha$. Somit ist
  $C\cap\alpha$ unbeschränkt in $\alpha$. Also ist $\alpha\in C$, da
  $C$ abgeschlossen in $\kappa$. Wegen $\alpha$ regulär ist $C\cap
  E\neq\emptyset$.
\end{itemize}
\hfill $\square$

{\bf Lemma 8:} Sei $W$ ein inneres Modell von $\ZFC$. Sei $\kappa$
Mahlo. Dann ist $\kappa$ Mahlo in $W$.

{\bf Beweis:} Nach Lemma 5 ist $\kappa$ unerreichbar in $W$. Sei $C\in
W$ abgeschlossen unbeschränkt in $\kappa$. Dann existiert $\alpha\in
C$ mit $\alpha$ regulär. Dann ist aber $\alpha$ auch regulär in
$W$. Somit ist $\{\alpha<\kappa|\alpha\mbox{ regulär in }W\}$
stationär in $\kappa$ (in $W$).

\hfill $\square$

Wir können die Definition von Mahlo ``iterieren'', zum Beispiel

{\bf Definition:} $\kappa$ ist $2$-Mahlo $:\Leftrightarrow$ $\kappa$
ist unerreicbar und $\{\alpha<\kappa|\alpha\mbox{ Mahlo}\}$ ist
stationär in $\kappa$.

{\bf Bemerkung:} $\kappa$ ist $2$-Mahlo $\Rightarrow$ $\kappa$ Mahlo.

{\bf Satz 9:} Es sind äquivalent:
\begin{itemize}
\item[(1)] $\kappa$ ist $2$-Mahlo
\item[(2)] für alle $A\subseteq V_\kappa$ existiert Mahlo
  $\alpha<\kappa$ mit $V_\alpha\prec\left<V_\kappa,A\right>$
\item[(3)] für alle $A\subseteq V_\kappa$ mit
  $\left<V_\kappa,A\right>\models\varphi$ ($\varphi$ erststufig)
  existiert Maho $\alpha<\kappa$ mit $\left<V_\alpha,A\cap
  V_\alpha\right>\models\varphi$.
\end{itemize}
{\bf Beweis:} Völlig analog zu Beweis von Satz 7.

\hfill $\square$

Noch größere Kardinalzahlen erhalten wir, wenn wir die Eigenschaft (3)
aus Satz 4 ``für alle $A\subseteq V_\kappa$ und erstsufigen $\varphi$
mit $\left<V_\kappa,A\right>\models\varphi$ existiert ein
$\alpha<\kappa$ mit $\left<V_\alpha,A\cap
V_\alpha\right>\models\varphi$'' wesentlich verstärken.

{\bf Definition:} $\kappa$ ist schwach kompakt
(bzw. $\Pi_1^1$-unbeschreibbar wenn folgendes gilt:

Sei $A\subseteq V_\kappa$ und es gelte für alle $B\subseteq V_\kappa$
dass $\left<V_\kappa, A, B\right>\models\varphi$ ($\varphi$
erststufig). dann existiert $\alpha<\kappa$ mit für alle $B\subseteq
V_\alpha$ $\left<V_\alpha,A\cap V_\alpha,B\right>\models\varphi$.

{\bf Definition:} Sei $Q$ eine Eigenschaft von Ordinalzahlen. Wir
sagen $Q$ ist eine $\Pi_1^1$-Eigenschaft, wenn existiert erststufiges
$\varphi$ mit $Q(\kappa)$ $\Leftrightarrow$ für alle $B\subseteq
V_\kappa$ gilt $\left<V_\kappa,B\right>\models\varphi$.

{\bf Bemerkung:} Sei $Q$ eine $\Pi_1^1$-Eigenschaft. Sei $\kappa$
schwach kompakt. Weiterhin gelte $Q(\kappa)$. Dann existiert ein
$\tau<\kappa$ mit $Q(\tau)$. ``$\kappa$ ist schwach kompakt'' ist also
keine $\Pi_1^1$-Eigenschaft. Die bisherigen Eigenschaften sind aber
$\Pi_1^1$-Eigenschaften.

{\bf Lemma 10:} ``$\kappa$ ist regulär'', ``$\kappa$ ist
unerreichbar'', ``$\kappa$ ist Mahlo'', ``$\kappa$ ist $2$-Mahlo''
sind $\Pi_1^1$-Eigenschaften.

{\bf Beweis:} Nehme einfach die üblichen Definitionen.

\hfill $\square$

{\bf Satz 11:}
\begin{itemize}
\item[(a)] $\kappa$ schwach kompakt $\Rightarrow$ $\kappa$ ist
  $2$-Mahlo
\item[(b)] $\kappa$ schwach kompakt $\Rightarrow$ existiert
  $\tau<\kappa$ mit $\tau$ ist $2$-Mahlo.
\end{itemize}

{\bf Beweis:}
\begin{itemize}
\item[(a)] Sei $\kappa$ schwach kompakt. Nach Definition ist dann die
  Bedingung (3) aus Satz 4 erfüllt. Also ist $\kappa$
  unerreichbar. Wir zeigen zuerst, dass $\kappa$ Mahlo ist. Hierzu
  zeigen wir die Bedingung (3) aus Satz 7. Sei also $A\subseteq
  V_\kappa$ mit $\left<V_\kappa,A\right>\models\varphi$ ($\varphi$
  erststufig). Zu zeigen ist, es existiert ein reguläres
  $\alpha<\kappa$ mit $\left<V_\alpha,A\cap
  V_\alpha\right>\models\varphi$. Nach Lemma 10 sei $\psi$ erststufig
  mit:
  $$ \tau\mbox{ ist regulär}\Leftrightarrow\mbox{für alle }B\subseteq
  V_\kappa\left<V_\kappa,B\right>\models\psi$$
  Da ja $\kappa$ regulär ist, gilt dann:
  $$ \mbox{für alle }B\subseteq
  V_\kappa\phantom{a}\left<V_\kappa,A,B\right>\models\varphi\wedge\psi$$
  Also wegen $\kappa$ schwach kompakt existiert $\alpha<\kappa$ mit
  $$ \mbox{für alle }B\subseteq
  V_\alpha\phantom{a}\left<V_\alpha,A\cap
  V_\alpha,B\right>\models\varphi\wedge\psi$$ Also ist $\alpha$
  regulär und $\left<V_\alpha,A\cap V_\alpha\right>\models\varphi$,
  was zu zeigen war. Nun wissen wir, dass $\kappa$ Mahlo ist. Nun ist
  aber nach Lemma 10 auch die Eigenschaft ``$\tau$ ist Mahlo'' eine
  $\Pi_1^1$-Eigenschaft. Also können wir völlig Analog die Eigenschaft
  (3) aus Satz 9 zeigen. Somit ist $\kappa$ $2$-Mahlo.
\item[(b)] Nach Lemma 10 ist auch die Eigenschaft ``$\tau$ ist
  2-Mahlo'' eine $\Pi_1^1$-Eigenschaft. Also folgt die Behauptung aus
  (a) und Bemerkung.
\end{itemize}

\hfill $\square$

{\bf Lemma 12:} Sei $\kappa$ schwach kompakt. Seien
$A_1,\ldots,A_n\subseteq V_\kappa$, $\varphi$ erststufig, und es
gelte: Für alle $B\subseteq V_\kappa$ haben wir
$\left<V_\kappa,A_1,\ldots,A_n\right>\models\varphi$. Dann existiert
ein $\alpha<\kappa$ mit für alle $B\subseteq V_\alpha$ gilt
$\left<V_\alpha,A_1\cap V_\alpha,\ldots,A_n\cap
V_\alpha,B\right>\models\varphi$.

{\bf Beweisskizze:} Setze $A=\bigcup_{i=1}^{n}\{i\}\times A_i$. dann
ist $A_i$ kanonisch definierbar in $\left<V_\kappa,A\right>$ durch
$A_i=\{x|\left<i,x\right>\in A\}$. Übersetze hiermit $\varphi$ in
$\varphi^x$. Wähle $\alpha<\kappa$ mit für alle $B\subseteq
V_\alpha\phantom{a}\left<V_\alpha,A\cap
V_\alpha,B\right>\models\varphi^x \wedge
\forall_{\gamma\in\On}\exists_{\delta\in\On}\gamma\in\delta$. Dann ist
$\alpha$ Limesordinalzahl, also $A_i\cap
V_\alpha=\{x|\left<i,x\right>\in A\cap V_\alpha\}$. Durch
Rückübersetzung also für alle $B\subseteq V_\alpha$
$\left<V_\alpha,A_1\cap V_\alpha,\ldots,A_n\cap
V_\alpha,B\right>\models\varphi$

\hfill $\square$

{\bf Lemma 13:} Sei $\kappa$ schwach kompakt. Sei
$A_1,\ldots,A_n\subseteq V_\kappa$, $\varphi$ erststufig, und es
gelte: für alle $B\subseteq V_\kappa$ ist
$\left<V_\kappa,A_1,\ldots,A_n,B\right>\models\varphi$. Setze
$E=\{\alpha<\kappa|\mbox{für alle }B\subseteq V_\alpha\mbox{ gilt
}\left<V_\alpha,A_1\cap V_\alpha,\ldots,A_n\cap V_\alpha,
B\right>\models\varphi\}$. Dann ist $E$ stationär in $\kappa$.

{\bf Beweis:} Sei $C\subseteq\kappa$ abgeschlossen unbeschränkt in
$\kappa$. Zu zeigen $E\cap C\neq\emptyset$. Nun aber: für alle
$B\subseteq V_\kappa$
$\left<V_\kappa,A_1,\ldots,A_n,C,B\right>\models\underbrace{\varphi\wedge\forall_{\gamma\in\On}\exists_{\delta\in
    C}\gamma<\delta}_\psi$. Also nach Lemma 12 existiert
$\alpha<\kappa$ mit: für alle $B\subseteq
V_\alpha\phantom{a}\left<V_\alpha,A_1\cap V_\alpha,\ldots,A_n\cap
V_\alpha,C\cap V_\alpha, B\right>\models\psi$. Dann natürlich
$\alpha\in E$. Aber auch $\sup(C\cap\alpha)=\alpha$. Also $\alpha\in
C$, da $C$ abgeschlossen in $\kappa$.

\hfill $\square$

{\bf Definition:} Sei $f:[z]^n\rightarrow\gamma$ eine Partition von
$[z]^n$. $x\subseteq z$ heißt {\it homogen} für $f$ genau dann wenn
$\exists_{\beta<\gamma} f''[x]^n\subseteq\{\beta\}$.
$$ \kappa \rightarrow (\delta)_\gamma^n \Leftrightarrow \mbox{für alle
} f:[\kappa]^n\rightarrow\gamma\mbox{ existiert
}X\subseteq\kappa\mbox{ homogen für }f\mbox{ mit }\otp(X)=\delta $$

{\bf Bemerkung:} Falls $\kappa\rightarrow(\delta)_\gamma^n$ und
$\overline{\kappa}\ge\kappa,\lim(\kappa),\overline{\delta}\le\delta,\overline{\kappa}\le\kappa$,
so
$\overline{\kappa}\rightarrow(\overline{\delta})_{\overline{\gamma}}^{\overline{n}}$.

{\bf Wiederholung:}
Folgende Begriffe werden als bekannt vorausgesetzt:
\begin{itemize}
\item Ein {\it Baum} ist eine partiell geordnete Menge, sodass die
  Vorgänger jedes Elementes wohlgeordnet sind. Äquivalent dazu, es ist
  eine global fundierte partielle Ordnung und die Vorgänger jedes
  Elements sind total geordnet. Für einen Baum $T$ sind $T_\alpha$ die
  Elemente sodass die Vorgänger den Ordnungstyp $\alpha$ haben. Die
  Höhe des Baumes ist die kleinste Ordinalzahl $\alpha$ derart dass
  $T_\alpha$ leer ist.
\item Ein {\it Filter} auf einer Menge $X$ ist eine Menge von
  Teilmengen von $X$ die die folgenden Eigenschaften hat: $X$ liegt
  drin, $\emptyset$ liegt nicht drin, mit je zwei Elementen ist der
  Durchschnitt wieder drin, jede Obermenge einer Menge die drin ist
  ist wieder drin. Beispiel: Die Menge der Teilmengen von $[0;1]$ die
  das Maß $1$ haben sind ein Filter.
\item Ein {\it Ultrafilter} ist ein bezüglich Inklusion maximaler
  Filter. Ein Filter auf einer Menge $X$ ist genau dann ein
  Ultrafilter wenn für jede Teilmenge $A\subseteq X$ gilt entweder $A$
  oder $X\backslash A$ ist in dem Filter.
\end{itemize}

{\bf Definition:} Ein $\kappa$-Baum ist ein Baum der Höhe $\kappa$ mit
$\forall_{\alpha<\kappa} |T_\alpha|<\kappa$.

{\bf Definition:} Sei $\mathfrak{F}$ ein Filter auf $X$.
\begin{itemize}
  \item[(a)] $\mathfrak{F}$ ist {\it $\kappa$-vollständig}, wenn gilt:
    $(\mathfrak{G}\subseteq\mathfrak{F}\mbox{ und }
    0<|\mathfrak{G}|<\kappa)\Rightarrow\bigcap\mathfrak{G}\in\mathfrak{F}$.
  \item[(b)] $\mathfrak{F}$ ist {\it nichttrivial}, wenn gilt:
    $\forall_{a\in X} X\backslash\{a\}\in\mathfrak{F}$.
\end{itemize}

{\bf Beispiel:} Ist $\kappa>\omega$ regulär, so ist
$\varphi_\kappa=\{A\subseteq\kappa|A\supseteq C\mbox{ für ein
  abg. unb. }C\subseteq\kappa\}$ nichttrivialer $\kappa$-vollständiger
Filter auf $\kappa$.

{\it ``Wir kommen jetzt zum ersten Hauptsatz, der etwas komplizierter
  wird, und wo uns der Beweis auch mehrere Sitzungen lang beschäftigen
  wird''}

{\bf Satz 14:} Sei $\kappa$ Kardinalzahl. Dann sind äquivalent:
\begin{itemize}
\item[(1)] $\kappa$ ist schwach kompakt
\item[(2)] $\kappa\rightarrow(\kappa)_2^2$ und $k>\omega$
\item[(3)] $\kappa$ ist unerreichbar und jeder $\kappa$-Baum besitzt
  einen Zweig der Länge $\kappa$
\item[(4)] $\kappa>\omega$, und es gilt: falls
  $\mathfrak{H}\subseteq\mathfrak{P}(\kappa)$ mit
  $|\mathfrak{H}|\le\kappa$, so existiert nichttrivaler
  $\kappa$-vollständiger Filter $\mathfrak{F}$ auf $\kappa$ mit
  $\forall_{A\in\mathfrak{H}}(A\in\mathfrak{F}\mbox{ oder
}\kappa\backslash A\in\mathfrak{F})$.
\item[(5)] $\kappa$ ist unerreichbar und es gilt: falls $M$ transitiv,
  $M\models\ZFC^-$, $\kappa\in M$, $|M|=\kappa$, so existiert
  $j:M\rightarrow_{\Sigma_\omega} N$ mit $N$ transitiv,
  $j\upharpoonright\kappa=\id\upharpoonright\kappa$,
  $j(\kappa)>\kappa$.
\item[(6)] Für alle $A\subseteq V_\kappa$ existiert transitives $M$
  und $B\subseteq M$ mit
  $\left<V_\kappa,A\right>\prec\left<M,B\right>$, $M\neq V_\kappa$.
\end{itemize}

{\bf Beweis:}
\begin{itemize}
  \item (1)$\Rightarrow$(2): Nur zu zeigen
    $\kappa\rightarrow(\kappa)_2^2$. Sei also $f:[\kappa]^2\rightarrow
    2$. Für $\alpha<\kappa$ definiere eine monotone Folge
    $\left<\gamma_\xi^\alpha|\xi<\delta_\alpha\right>$\footnote{$\delta_\alpha$
      selbst wird erst durch die Rekursion festgelegt.} rekursiv
    durch: Sei $\left<\gamma_\xi^\alpha|\xi<\eta\right>$ schon
    konstruiert. Setze
    $B_\eta^\alpha=\{\gamma<\alpha|\forall_{\xi<\eta}
    (\gamma^\alpha_\xi<\gamma\mbox{ und
    }f(\{\gamma_\xi^\alpha,\gamma\})=0)\}$. Falls
    $B_\eta^\alpha\neq\emptyset$, setze $\gamma_\eta^\alpha=\min
    B_\eta^\alpha$. Andernfalls sei $\delta_\alpha=\eta$. Setze
    $h(\alpha)=\sup\{\gamma_\xi^\alpha|\xi<\delta_\alpha\}\le\alpha$
    und $g(\alpha)=\left<\gamma_\xi^\alpha|\xi<\delta_\alpha\right>$.
    \begin{itemize}
      \item 1. Fall: $E=\{\alpha<\kappa|h(\alpha)<\alpha\}$ ist
        stationär. Dann nach Fodor\footnote{\href
          {https://de.wikipedia.org/wiki/Satz\_von\_Fodor} {Satz aus
            der Mengenlehre}} existiert stationäres $E_0\subseteq E$
        mit $h\upharpoonright E_0$ konstant. Wegen $\kappa$
        unerreichbar existiert dann $E_1\subseteq E_0$ mit
        $|E_1|=\kappa$ und $g\upharpoonright E_1$ konstant. Nach
        Konstruktion aber dann für alle
        $\{\alpha,\beta\}\in[E_1]^2\phantom{a}f(\{\alpha,\beta\})=1$. Also
        hat $f$ homogene Menge der Mächtigkeit $\kappa$.
      \item 2. Fall: es existiert abgeschlossen unbeschränktes
        $C\subseteq\kappa$ mit $\forall_{\alpha\in C}
        h(\alpha)=\alpha$. Zeige: es existiert unbeschränktes
        $B\subseteq\kappa$ mit $\forall_{a\in[B]^2}f(a)=0$. Angenommen
        nicht. Dann gilt für alle $B\subseteq
        V_\kappa\phantom{a}\left<
        V_\kappa,f,B\right>\models\underbrace{(B\subseteq\On\mbox{ und
          }\forall_{\gamma\in\On}\exists_{\delta\in
            B}\gamma<\delta\Rightarrow\exists_{\eta,\rho\in B}
          f(\{\eta,\rho\})=1)}_\varphi$. Also existiert nach Lemma 13
        wegen $\kappa$ schwach kompakt ein $\alpha\in C$ mit für alle
        $B\subseteq V_\alpha\phantom{a}\left<V_\alpha,f\cap
        V_\alpha,B\right>\models\varphi$. Dies ist ein Widerspruch, da
        $B=\{\gamma_\xi^\alpha|\xi<\delta_\alpha\}$ ein Gegenbeispiel
        ist.
    \end{itemize}
    \item (2)$\Rightarrow$(3): Zeige zuerst: $\kappa$ ist
      unerreichbar. $\kappa$ ist regulär, denn angenommen
      $\kappa=\sup_{i<\tau}\kappa_i$ mit
      $\left<\kappa_i|i<\tau\right>$ normal\footnote{streng monoton
        und stetig}, $\kappa_0=0$, $\tau<\kappa$. Definiere
      $f:[k]^2\rightarrow 2$ durch $f(\{\alpha,\beta\})=0
      \Leftrightarrow\sup\{i|\kappa_i\le\alpha\}=
      \sup\{i|\kappa_i\le\beta\}$ liefert Widerspruch. \newline Sei
      $\tau<\kappa$. Zu zeigen $2^\tau\le\kappa$. Gilt wegen
      $2^\tau\not\rightarrow(\tau^+)_2^2$. Hierzu definiere $F:[^\tau
        2]^2\rightarrow 2$\footnote{$^\tau 2$ ist die Menge der
        Funktionen von $\tau$ nach $2$} wie folgt. Sei $\prec$ eine
      Wohlordnung auf $^\tau 2$ und $<_\ell$ die lexikographische
      Ordnung auf $^\tau 2$. Setze dann
      $F(\{g,h\})=0\Leftrightarrow(g\prec h\mbox{ und }h<_\ell
      h)$. Dies tut es, da die lexikographische Ordnung die
      Eigenschaft hat, dass es keine aufsteigende oder absteigende
      Folge der Länge $\tau^+$ gibt (siehe späteres Argument).
    \item (2)$\Rightarrow$(3): Sei $\utilde{T}=\left<T,\le_T\right>$
      ein $\kappa$-Baum. Ohne Einschränkung $T=\kappa$. Für $t\in
      T_\alpha$ sei $b_t=\left<t_\gamma|\gamma\le\alpha\right>$ die
      bezüglich $\le_T$ aufsteigende fFolge der $\le_T$-Vorgänger von
      $t$. Definiere $f:[\kappa]^2\rightarrow 2$ durch
      $f(\{\overline{t},t\})=0\Leftrightarrow(\overline{t}<t\mbox{ und
      }b_{\overline{t}}<_\ell b_t)$. Hierbei:
      $b<_\ell\overline{b}\Leftrightarrow(b\subsetneq\overline{b}\mbox{
        oder }b(\gamma)<\overline{b}(gamma)\mbox{, wobei
      }\gamma=\min\{\mu|b(\mu)\neq\overline{b}(\mu)\})$. Gemäß (2)
      existiert $X\subseteq\kappa$ mit $|X|=\kappa$ und $i<2$ mit
      $f''[X]^2=\{i\}$. Setze $R_0=\le_\ell$, $R_1=_\ell\ge$,
      $\overline{R}_0=\le$, $\overline{R}_1=\ge$. Also gilt:
      $\gamma,\delta\in X,\gamma<\delta\rightarrow b_\gamma R_i
      b_\delta$. Sei nun $\left<t_\delta,\delta<\kappa\right>$ die
      monotone Aufzählung von $X$. Wegen $\forall_{\alpha<\kappa}
      |T_\alpha|<\kappa$ können wir ohne Einschränkung annehmen, dass
      $\forall_{\delta<\kappa}\height(t_\delta)\ge\delta$. Für
      $\gamma<\kappa$ setze $a_\gamma= \left<b_{t_\delta}(\gamma)|
      \delta\in\kappa\backslash\gamma\right>$. Zeige durch Induktion
      über $\gamma<\kappa$: \newline (*) $a_\gamma$ ist schließlich
      konstant.\newline Für $\beta<\gamma$ wähle hierzu
      $\beta\le\mu_\beta<\kappa$ mit $\left<\beta_{t_\delta}(\beta)|
      \delta\in\kappa\backslash\mu_\beta\right>$ konstant ist. Sei
      $\mu=\sup_{\beta<\gamma}\mu_\beta$. Dann ist
      $\left<\beta_{t_\delta}\upharpoonright\gamma|
      \delta\in\kappa\backslash\mu\right>$ konstant. Nach Definition
      von $<_\ell$ gilt daher: $\mu\le\delta<\eta<\kappa\rightarrow
      b_{t_\delta}(\gamma)\overline{R}_i b_{t_\eta}(\gamma)$. Ist aber
      $i=0$, so ist also
      $\left<b_{t_\delta}(\gamma)|\delta\in\kappa\backslash\mu\right>$
      schwach monoton steigend, also schließlich konstant, da
      $|T_\gamma|<\kappa$ und $\kappa$ regulär. Ist aber $i=1$, so ist
      $\left<b_{t_\delta}(\gamma)|\delta\in\kappa\backslash\gamma\right>$
      schwach monoton fallend, also schließlich konstant. Für
      $\gamma<\kappa$ setze nun
      $r_\gamma=\mbox{``}\lim\mbox{\textquotedblright} a_\gamma \in
      T_\gamma$. Dann ist $\{r_\gamma|\gamma<\kappa\}$ Zweig der Länge
      $\kappa$ in T.
    \item (3)$\Rightarrow$(4): Sei
      $\mathfrak{H}\subseteq\mathfrak{P}(\kappa)$ mit
      $|\mathfrak{H}|\le\kappa$. Sei
      $\mathfrak{H}\cup\{\kappa\}=\{A_\alpha|\alpha<\kappa\}$. Für
      $\alpha<\kappa$ setze $B_\alpha^0=A_\alpha$,
      $B_\alpha^1=\kappa\backslash A_\alpha$. Setze
      $T=\{p:\gamma\rightarrow 2|\gamma<\kappa,
      |\bigcup\limits_{\alpha<\gamma}B_\alpha^{p(\alpha)}|=\kappa\}$,
      $\utilde{T}=\left<T,\subseteq\right>$. Beachte, dass $p\in
      T\mbox{ und }\beta<\dom(p)\Rightarrow p\upharpoonright\beta\in
      T$. $\utilde{T}$ ist $\kappa$-Baum, denn:
      \begin{itemize}
      \item[(i)] Höhe von $\utilde{T}$ ist $\le\kappa$.
      \item[(ii)] für $\gamma<kappa$ ist $|T_\gamma|<\kappa$, denn:
        \vspace{20pt}$T_\gamma=\{p\in T|\dom(p)=\gamma\}$, also $|T_\gamma|\le
        2^{|\gamma|}\mathnode{l1}{<}\kappa \mathnode{l2}{{\phantom{aaa\mbox{daa}}\kappa\mbox{ unerreichbar}}}$.
        \item[(iii)] Höhe von $\utilde{T}$ ist $\ge\kappa$.
          \begin{tikzpicture}[overlay]
            \path [>=stealth, <-, shorten <= 3pt, shorten >=3 pt]
            (l1) edge [bend left=70] (l2);
          \end{tikzpicture}
      \end{itemize}
      Für $\delta<\kappa$ definiere $p_\delta:\gamma\rightarrow 2$
      durch die Forderung $\delta\in
      B_\alpha^{p_\delta(\alpha)}$. Wegen $\kappa$ unerreichbar
      existiert $p$ mit $|\{\delta<\kappa|p_\delta=p\}|=\kappa$. Dann
      aber $p\in T_\gamma$. Das beweist (iii). Damit ist $\utilde{T}$
      ein $\kappa$-Baum. \newline Wegen (3) hat $\utilde{T}$ einen
      Zweig $b$ der Länge $\kappa$. Setze
      $\mathcal{F}=\{D\subseteq\kappa|\exists_{p\in
        b}\exists_{\delta<\kappa}
      D\supseteq(\bigcap\{B_\alpha^{p(\alpha)}|
      \alpha\in\dom(p)\}\backslash\delta)\}$. $\mathcal{F}$ ist wie
      gewünscht.
    \item (4)$\Rightarrow$(5): Zeige zuerst: $\kappa$ ist
      unerreichbar.
      \begin{itemize}
      \item[(i)] $\kappa$ ist regulär: Sei $\mathcal{F}$
        nichttrivialer $\kappa$-vollständiger Filter auf $\kappa$. Da
        $\forall_{\delta<\kappa}\kappa\backslash\delta=\bigcap\limits_{\gamma<\delta}\kappa\backslash\{\gamma\}\in\mathcal{F}$. Angenommen
        $\kappa$ sei singulär. Wähle also $B\subseteq\kappa$ konfinal
        mit $|B|<\kappa$. Dann ist
        $\emptyset=\bigcap\limits_{\delta\in
          B}\kappa\backslash\delta\in\mathcal{F}$. Widerspruch.
      \item[(ii)] Sei $\tau<\kappa$. Angenommen
        $2^\tau\ge\kappa$. Wähle dann $f_i:\tau\rightarrow 2$ für
        $i<\kappa$ mit $i\neq j\rightarrow f_i\neq f_j$. Für
        $\delta<\tau$ setze $A_\delta=\{i<\kappa|f_i(\delta)=0\}$ und
        sei $\mathfrak{H}=\{A_\delta|\delta<\tau\}$. Sei hierzu
        $\mathcal{F}$ wie in (4). Definiere $f:\tau\rightarrow 2$ so,
        dass
        $B_\delta=\{i<\kappa|f_i(\delta)=f(\delta)\}\in\mathcal{F}$. Setze
        $B=\bigcap\limits_{\delta<\tau}B_\delta\in\mathcal{F}$. Sei
        $i\in B$. Dann ist $f=f_i$. Also hat $B$ höchstens ein
        Element. Dies ist ein Widerspruch zur Nichttrivialität von
        $\mathcal{F}$.
      \end{itemize}
\end{itemize}

{\bf Übung 2:}

{\bf Aufgabe 1:} Sei $\kappa$ schwach kompakt. Seien
$D,E\subseteq\kappa$ stationär in $\kappa$. Man zeige: es existiert
reguläres $\alpha<\kappa$ mit $D\cap\alpha$,$E\cap\alpha$ stationär in
$\alpha$.

{\bf Aufgabe 2:} Sei $U$ ein Ultrafilter auf $X$. Weiterhin sei
$\kappa$ eine Kardinalzahl. Zeigen Sie, dass folgende Aussagen
äquivalent sind:
\begin{itemize}
\item[(1)] $U$ ist $\kappa$-vollständig
\item[(2)] Falls $\mathfrak{G}\subseteq U$, $|\mathfrak{G}|<\kappa$,
  so $\bigcap\mathfrak{G}\neq\emptyset$.
\item[(3)] Falls $\bigcup\mathfrak{H}\in U$, $|\mathfrak{H}|<\kappa$,
  so existiert $A\in\mathfrak{H}$ mit $A\in U$.
\end{itemize}

\end{document}
