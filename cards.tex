%% -*- coding: utf-8 -*

\documentclass[a4paper,fontsize=11pt]{scrartcl}
\usepackage[utf8]{inputenc}
\usepackage{ngerman}
\usepackage{CJKutf8}
\usepackage{graphicx}
\usepackage{wrapfig}
\usepackage{fancyhdr}
\usepackage{amsfonts}
\newenvironment{Japanese}{%  
\CJKfamily{min}%  
\CJKtilde  
\CJKnospace}{}
\pagestyle{fancy}

\fancyhf{}
\fancyhead[L]{Große Kardinalzahlen}
\fancyhead[C]{}
\fancyhead[R]{\thepage}
\renewcommand{\headrulewidth}{0.4pt}
%%\author{Prof. Dr. H.-D. Donder}       %Authors
\begin{document}
\flushleft
\section*{Vorwort}
{\bf Vorkenntnisse:} Logik, Modelle der Mengenlehre (Konstruktibles
Universum, Innere Modelle, kein Forcing)

{\bf Inhalt:} Ursprungliche Motivation für groke Kardinalzahl-Axiome
war, dass $\omega$ ein Sprung in den Kardinalzahlen ist. Man überlegte
ob so ein Sprungverhalten öfter vorkommen könnte. Die erste
Vorgehensweise war, bestimmte Eigenschaften von $\omega$ zu isolieren,
und dann eine Zahl mit den gleichen Eigenschaften ungleich $\omega$
gefordert hat. Dies ist allerdings inzwischen weitgehend obsolet.

Die Theorien die man durch große Kardinalzahlen bekommt ist eine
Verstärkung der ursprünglichen Theorie. Man kann mit immer neuen
großen Kardinalzahlen immer neue Aussagen über natürliche Zahlen
beweisen.

Das hier Behandelte ist im Wesentlichen die Einführung immer größerer
Kardinalzahlen, relevant wird immer sein, ob man noch die natürlichen
inneren Modelle für große Kardinalzahlen hat.

{\bf Literatur:}
\begin{itemize}
  \item Drake, Set Theory - An Introduction To Large Cardinals
  \item Jech, Set Theory
  \item Kanamori, The Higher Infinite
\end{itemize}

{\bf Notation:}
\begin{itemize}
  \item $[X]^n$ = Menge aller $n$-Elementigen Teilmengen von $X$
  \item $[X]^{<\omega} = \bigcup_n[X]^n$ = Menge aller endlichen
    Teilmengen von $X$
\end{itemize}

Wir betrachten oft Strukturen der Form
$\left<M,\in,A_1,\ldots,A_n\right>$ mit $A_i\subseteq M$.

{\bf Konvention:} Lasse das $\in$ weg, identifiziere
$\left<M,A_1,\ldots,A_n\right>$ mit
$\left<M,\in,A_1,\ldots,A_n\right>$.

{\bf Elementare Substruktur:} Sei
$\mathfrak{M}=\left<M,A_1,\ldots,A_n\right>$, $N\subseteq M$ und
$\mathfrak{N}=\left<N,A_1\cap N,\ldots,A_n\cap N\right>$ die zugehörige
Substruktur von $\mathfrak{M}$. Dann ist $\frak N$ elementare Substruktur
von $\frak M$ (Bez. $\mathfrak{N}\prec\mathfrak{M}$ genau dann wenn für alle
$\varphi$ und $a_1,\ldots,a_k\in N$
$$ \mathfrak{M}\models \phi[a_1,\ldots,a_k] \mbox{gdw} \mathfrak{N}\models\phi[a_1,\ldots,a_k]$$

{\bf Konvention:} Schreibe $N\prec\left<M,A_1,\ldots,A_n\right>$ statt $\left<N,A_1\cap N,\ldots,A_n\cap N\right>\prec\left<M,A_1,\ldots,A_n\right>$.

{\bf Skolemfunktionen:} Sei
$\mathfrak{M}=\left<M,\in,A_1,\ldots,A_n\right>$ eine
${\mathcal{L}}$-Struktur. Für eine $\mathcal{L}$-Formel
$\varphi(\vec{x},y)$ definiere $f_\varphi$ so, dass gilt

$$ \mathfrak{M}\models \exists_y \varphi[\vec{a},y] \Rightarrow \mathfrak{M}\models\varphi[\vec{a},f_\varphi(\vec{a})]$$

Setze
$\mathcal{F}=\{f_\phi|\phi\phantom{a}\mathcal{L}-\mbox{Formel}\}$. Dann
gilt:

Sei $N\subseteq M$, $N\neq\emptyset$, unter allen $f\in\mathcal{F}$
abgeschlossen (d.h. $\vec{a}\in N,f\in\mathcal{F}\Rightarrow
f(\vec{a})\in N$). Dann ist $N\prec\mathfrak{M}$.

\underline{Wir sagen:} $\mathcal{F}$ ist Menge von Skolemfunktionen
für $\mathfrak{M}$. Beachte, dass $\mathcal{F}$ abzählbar ist.

\end{document}
